\documentclass[10pt, a4paper, sans]{moderncv}
%\usepackage{FiraSans} 

% ModernCV Settings
%%%%%%%%%%%%%%%%%%%%%%%%%%%%%%%%%%%%%%%%%%%%%%%%%%%%%%%%%%%%%%%%%%%%%%%%%%%%%%%

\usepackage[scale=0.85]{geometry}
\usepackage{multicol}
\usepackage{enumitem}
\setlist[itemize]{noitemsep, topsep=0pt}
\setlength{\multicolsep}{2pt}

\moderncvtheme{fancy}
\moderncvcolor{grey}
\setlength{\hintscolumnwidth}{3.3cm}
\nopagenumbers{}


% Personal Information
%%%%%%%%%%%%%%%%%%%%%%%%%%%%%%%%%%%%%%%%%%%%%%%%%%%%%%%%%%%%%%%%%%%%%%%%%%%%%%%

\name{Jackson}{Campolattaro}

\address{22784 Portico Pl.}{Ashburn, VA}{20148, USA}
\email{jackcamp@vt.edu}
\phone[mobile]{(703) 772-1748}
\social[linkedin]{jacksoncampolattaro}
\social[github]{JacksonCampolattaro}

\quote{
    Self-motivated Computer Engineering student
    with programming experience and an enthusiasm for Open Source principles.\\
    I use complex, long-term personal projects as a medium for exploration 
    of new programming languages, tools, and techniques.
}

% Document Content
%%%%%%%%%%%%%%%%%%%%%%%%%%%%%%%%%%%%%%%%%%%%%%%%%%%%%%%%%%%%%%%%%%%%%%%%%%%%%%%

\begin{document}

\makecvtitle


\section{Education}
\cventry{Graduation Spring 2021}{Virginia Polytechnic}{Computer Engineering}{}{}{
    Pursuing a major in Computer Engineering
    with a minor and specialization in Computer Science.
    117 Credit Hours Earned.
    Expected to graduate 1 year early due to accelerated classes.
}
\section{Skills}

\subsection{Languages}
\cventry{6 Years Experience}{C++}{}{}{}{
    Libraries:
    Catch2, libsigc++, OpenMP, Intel TBB, Posix Threads,
    Gtkmm, Qt, OpenGL, GLFW, Magnum, CLI11, spdlog, Cereal,
    TOML11, Libsoundio, FFTW
}
\cventry{2 Years Experience}{C}{}{}{}{
    Libraries:
    Jansson, LibJWT
}
\cventry{2 Years Experience}{Python}{}{}{}{
    Libraries:
    OpenCV
}
\cventry{In Order of Experience}{Others}{}{}{}{
    Java, Rust, Verilog, HTML + CSS / Sass, Octave / Matlab, LabView, MIPS Assembly, x86 Assembly
}

\subsection{Tools}
\cvitem{}{
    \begin{multicols}{4}
        \begin{itemize}
            \item[] Git
            \item[] Linux
            \item[] Valgrind
            \item[] GDB
            \item[] Perf
            \item[] Travis CI
            \item[] Github Actions
            \item[] Ansible
            \item[] Doxygen
            \item[] Markdown
            \item[] \LaTeX
        \end{itemize}
    \end{multicols}
}


\section{Experience}

\subsection{Employment}
\cventry{May 2020--Present}{Google Summer of Code Apprentice}{CGAL}{}{}{
    Working remotely with a mentor in France to develop a new software package.
    The project is an Octree data structure, used in other packages.
    Required a mix of working with legacy code and creating entirely new code.
}
\cventry{June 2019--August 2019}{Innovation Committee Member}{Telos Corporation}{}{}{
    Worked in a 7 person group of interns
    researching the viability of future software security products.
    Built the frontend of a replacement for Telos’ employee intranet solution.
}

\subsection{Projects}
\cventry{August 2020--Present}{Quarter ID}{Python}{}{}{
    Leading a small team of interdisceplenary engineering students
    to develop a solution which determines the value of collectible coins using machine vision.
    Involves industrial imaging and lighting hardware,
    paired with bespoke software written in Python using OpenCV.
}
\cventry{July 2018--Present}{N-Body}{C++}{}{}{
    Building a multi-threaded dynamical simulation tool to improve my
    familiarity with optimization, build tools, design patterns,
    and libraries.
    Incorporated concepts including concurrency, event-driven programming,
    serialization, cache-optimization, and tree algorithms among others.
}
\cventry{August 2020--December 2020}{Spectrogram}{C++}{}{}{
    Developed a low-latency Spectrogram audio frequency visualizer alongside two other students.
    Involved navigating real-time limitations in a contemporary event-driven desktop application,
    as well as CI, build system engineering, and other team management logistics.
}

\end{document}

